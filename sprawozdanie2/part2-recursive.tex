%%%%%%%%%%%%%%%%%% Część II %%%%%%%%%%%%%%%%%

\section{Rekurencyjne sieci neuronowe}

\subsection{ Kodowanie sieci neuronowych o dowolnej topologii}

\begin{itemize}
\item \textbf{
Napisz w dowolnym języku/narzędziu sparametryzowany generator warstwowych-sieci-feed-forward albo każdy-z-każdym dla f0 lub f1.}	
\end{itemize}

\subsection{ Optymalizacja wag i topologii }

\begin{enumerate}
\item \textbf{Mutacja.}
	\begin{itemize}
		\item Ustaw parametry mutacji (Experiment->Genetics) w f0 i f1 tak, żeby dotyczyły tylko dodawania/usuwania neuronów i dodawania/usuwania połączeń. Ustaw "Neurons to add" na jeden rodzaj neuronu – Nu.
		\item Stwórz kilkanaście razy sekwencję 200 mutantów zaczynając od sieci z jednym neuronem. Co można powiedzieć o tych sekwencjach? Pomocniczo zrób wykresy liczby neuronów i połączeń neuronowych w funkcji n. 
W konsoli:
		\begin{verbatim}
		//make mutant sequence.
//ensure there is one genotype in the gene pool! 
var n,ile=200;
GenePools.group=0; //select first group
for(n=1;n<=ile;n++)
{
   GenePools.genotype=n-1; //select n-th genotype as ancestor
   GenePools.mutateSelected();
   Genotype.name="mutant "+n; //set its name to consecutive number
   GenePools.copySelected(0);
   //Simulator.print(""+n+" "+Genotype.nnsiz);
}
		\end{verbatim}
		
	\end{itemize}
\item \textbf{Krzyżowanie}
	\begin{itemize}
		\item Wybierz jedną z reprezentacji: f0 lub f1.
		\item Stwórz dwie sieci neuronowe o tej samej topologii, różniące się tylko wagami.
		\item Dokonaj ich krzyżowania. Czy rezultat spełnia postulaty dobrego krzyżowania podane w poprzednim ćwiczeniu? 
		\\W konsoli:
				\begin{verbatim}
//cross over two neural networks.
//ensure there are two genotypes in the gene pool! 
GenePools.group=0; //select first group
GenePools.genotype=0; //select first genotype
GenePools.crossoverSelected(1); //cross over with the second genotype
Genotype.name="child"; //set descendant's name
GenePools.copySelected(0);
}
		\end{verbatim}
		\item Powtórz tę operację. Czy krzyżowanie jest deterministyczne?
		\item Powtórz to zadanie (krzyżowanie) dla pary sieci neuronowych o bardzo różnych topologiach.
	\end{itemize}
\end{enumerate}