%%%%%%%%%%%%%%%%%% Część III %%%%%%%%%%%%%%%%%

\section{Uczenie nienadzorowane warstwowych sieci neuronowych}

\subsection{Projektowanie i testowanie prostej sieci typu SOM}

\begin{enumerate}
\item \textbf{Utwórz sieć typu mapa odwzorowania cech istotnych (SOM) o wymiarach 5x4 (sugerowanych przez StatsticaNN) dla zbioru IRIS: File|New|Network, potem wybierz Kohonen, Advise, Create.}

\item \textbf{
Otwórz edytor sieci (Edit|Network). Jakiego typu neurony są stosowane w warstwie wyjściowej (okienko z ilustracją sieci to sugeruje) ?}

\item \textbf{
Zwróć uwagę, jak zdefiniowany jest błąd popełniany przez sieć (Error function) (Suma po wszystkich przykładach odległości pomiędzy przykładem a najbliższym mu wektorem wag neuronów w warstwie wyjściowej => uczenie nienadzorowane!)}

\item \textbf{
Zobacz, jak można zmieniać topologię sieci manewrując parametrem Width dla drugiej warstwy.}

\item \textbf{
Naucz sieć (Train|Kohonen) przy domyślnych ustawieniach parametrów. Zwróć uwagę, że mimo nienadzorowanego charakteru uczenia ma sens wyświetlanie przebiegu błędu.}

\item \textbf{
Co więcej, można używać zbioru weryfikującego i oprzeć na nim warunek stopu. Zbadaj, jak uczy się sieć np. z warunkiem stopu Minimum improvement|Verification = 0.01, Window = 10 (wydaje się to sensowne, bo widać, że minimum lokalne znajdowane jest bardzo szybko i nie ma potrzeby uczyć sieci aż przez 100 epok).}

\item \textbf{
Testowanie sieci. Obejrzyj najpierw, jak „odpowiadają” poszczególne neurony na kolejne przykłady ze zbioru uczącego (Run|Activations). Potem obejrzyj odpowiedzi sieci w diagramie, który zachowuje jej topologię (Run|Topological Map).  Jak grupują się przykłady ?
Zwróć uwagę, że podczas przeglądania przykładów mapą topologiczną, można samemu nazywać przykłady (górne pole bez nazwy) i/lub neurony (dolne pole bez nazwy); działa także prawy przycisk myszki.}

\item \textbf{
Poza tym można też obejrzeć statystykę „wygrywania” współzawodnictwa przez poszczególne neurony w poszczególnych podzbiorach (Statistics|Win frequencies).}

\item \textbf{
W podobny sposób naucz i przeanalizuj dla tego samego zbioru sieć o innej topologii (szczególnie ciekawe jest to dla sieci o topologii „jednowymiarowej”, gdzie warstwa wyjściowa ma np. 20 neuronów ułożonych w rzędzie, parametr Width = 1 dla warstwy 2 w edytorze sieci).}
	
\end{enumerate}

\subsection{ Analiza rzeczywistego problemu przy użyciu sieci SOM}

 \paragraph{\textbf{Dane: Zbiór PROTEIN.STA, dostępny lokalnie w katalogu Statistica (Examples). Opisuje on spożycie protein różnego pochodzenia w wybranych krajach Europy. Dane nie są podzielone na klasy decyzyjne.}}
 \paragraph{\textbf{Zbuduj sieć dla tego problemu (sugerowany rozmiar warstwy wyjściowej 4x4 lub nawet 3x3, bo mało przykładów). Naucz ją i przetestuj. Czy da się wyciągnąć i uzasadnić jakieś wnioski dotyczące podobieństwa poszczególnych krajów pod względem spożycia protein? Można, podobnie jak dla IRIS, spróbować z siecią SOM jednowymiarową.}}


\subsection{Projektowanie i testowanie prostej sieci typu SOM}
 \paragraph{\textbf{Zobacz, co się dzieje, gdy opcja "Shuffle cases" jest wyłączona. Sprawdź wpływ parametrów algorytmu uczącego (blisko "skrajnych" wartości parametrów) na przebieg i efekty uczenia.}}