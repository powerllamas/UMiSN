%%%%%%%%%%%%%%%%%% Część I %%%%%%%%%%%%%%%%%

\section{Generowanie drzew decyzyjnych}

\subsection{Generowanie drzewa}

\begin{itemize}
\item \textbf{Obejrzyj zawartość plików \emph{golf.nam}, \emph{golf.dat} i~\emph{golf.tst}; ile przykładów zawiera zbiór uczący? Iloma atrybutami są~one opisane?}
	\\Zbiór uczący zawiera 14 przykładów. Są one opisane pięcioma atrybutami, w tym jednym atrybutem decyzyjnym.

\item \textbf{Wygeneruj drzewo dla zbioru przykładów \emph{golf}; ustawienia standardowe.}

Drzewo przed pruningiem:	 
	\begin{verbatim}
outlook = overcast: Play (4.0)
outlook = sunny:
|   humidity <= 75 : Play (2.0)
|   humidity > 75 : Don't Play (3.0)
outlook = rain:
|   windy = true: Don't Play (2.0)
|   windy = false: Play (3.0)
	\end{verbatim}
	
Drzewo po pruningu:
	\begin{verbatim}
outlook = overcast: Play (4.0/1.2)
outlook = sunny:
|   humidity <= 75 : Play (2.0/1.0)
|   humidity > 75 : Don't Play (3.0/1.1)
outlook = rain:
|   windy = true: Don't Play (2.0/1.0)
|   windy = false: Play (3.0/1.1)
	\end{verbatim}





\item \textbf{Przeanalizuj wyniki; czy udało się przeprowadzić pruning?}
	\\Nie udało się przeprowadzić pruningu. Drzewa są identyczne.

\item \textbf{Obejrzyj drzewo; ile ma węzłów decyzyjnych, a~ile liści?}
	\\Wygenerowane drzewo ma 3 węzły decyzyjne i 5 liści.

\item \textbf{ Prześledź ścieżkę od korzenia do wybranego liścia.}
		\\W korzeniu znajduje się test związany z atrybutem outlook. Jeśli dla klasyfikowanego przykładu wartość atrybutu outlook wynosi "sunny", to następny węzeł na ścieżce to ten związany z atrybutem humidity. Zakłądając, że dla omawianego przykładu wilgotność jest większ niż $ 75\% $ , liściem w analizowanej ścieżce będzie liść odnoszący się decyzji "Don't Play".
		\\Inaczej mówiąc, ścieżk ta klasyfikuje wszystkie dni w które jest słoneczna pogoda i wilgotność większa niż $ 75\% $ jako dni, w które nie gra się w golfa.
		
\item \textbf{Porównaj estymaty błędu dla drzewa oryginalnego (\emph{Unpruned}) i~uproszczonego (\emph{Pruned}).}
	\\Estymata błędu dla oryginalnego drzewa wynosi  $ 0\% $ a dla uproszczonego $ 38,5\% $ .

\item \textbf{Obejrzyj macierz pomyłek.}
	\\Ponieważ drzewo oryginalne nie generuje błędów dla zbioru uczącego, to też jedyne co można z niego odczytać, to właśnie, że żadne obiekty nie są błędnie klasyfikowane, oraz że w przypadku 9 przykładów drzewo poprawnie zaklasyfikowało przykłady do klasy "Play" a w 5 przypadkach do klasy "Don't Play".

\end{itemize}

\subsection{Konsultowanie}

\begin{itemize}
\item Dokonaj konsultacji wymyślonego przykładu dla wygenerowanego drzewa.
	\\Dokonano konsultacji przykładu o wartościach atrybutów: outlook = overcast, temperature = 5, humidity = 100, windy = true. W wyniku konsultacji podanego przykładu otrzymano decyzję "Play". Poneważ wartości atrybutów zostały podane w sposób deterministyczny to decyzja ta była pewna (prawdopodobieństwo, że analizowany przykład należy do klasy "Play" wynosiło 1). Wynik jest zgodny z oczekiwaniami, poniważ wygenerowane drzewo decyzyjne na podstawi galęzi "Outlook = overcast" przydziela wszystkie przykłady z taką wartością atrybutu do klasy "Don't play", niezależnie od wartości pozostałych atrybutów.
	
\item Konsultowanie przykładu ,,niepełnego''; dokładnie przeanalizuj wynik.
	\\Przykład niepełny miał postać: (overcast = sunny, temperature = 15, humidity = ?, windy = false). Przechodząc ścieżkę klasyfikującą ten przykład dochodzimy do atrybutu humidity. Nie jest znana wartość analizowanego przykładu na tym atrybucie, więc zostają obliczone prawdopodobieństwa przyjęcia konkretnych wartości. W analizowanym węźle (za gałęzią "sunny") znajduje się 5 przykładów uczących, w tym 2 ( $ 40\% $ ) przyjmują wartość "humidity" większą niż $ 75\% $ a 3  $ ( 60\% ) $ mniejszą. Dlatego przyjmuje się, że analizowany przykład będzie miał z prawdopodobieństwem $ 60\% $ wartość humidity większą niż $ 75\% $ a mniejszą z prawdopodobieństwem $ 40\% $ .	Dalej w drzewie sa juz tylko liście, można więc obliczyć rpawdopodobieństwo całych kompletów wartości: z prawdopodobieństwem $ 40\% $ analizowany przykład przyjmie wartości (overcast = sunny, temperature = 15, humidity  $ > 75\% $, windy = false) a z prawdopodobieństwem $ 60\% $ wartości  (overcast = sunny, temperature = 15, humidity $ >  75\% $ , windy = false). Drugie prawdopodobieństwo jest wyższe, więc z prawdopodobieństwem $ 60\% $ konsultowany przykład zostaje przypisany do klasy "Don't Play"
	
\item Konsultowanie, gdy znany jest rozkład prawdopodobieństwa.
	\\W przypadku przykładu ze znanym rozkładem prawdopodobieństwa analizowane są wszystkie gałęzie o niezerowym prawdopodobieństwie z wagami odpowiadającymi przypisanym danm wartośćiom prawdopodobieństwom. Analizowany przykład miał postać: P(sunny) = $ 0.8 $ , P(overcast) = $ 0.2 $ . Dlatego, jeśli wartość humidity była większa niż $ 75\% $ to algorytm przydzielał przykład z $ 20\% $ prawdopodobieństwem do klasy "Play" i $ 80 \% $ prawdopodobieństwem do klasy "Don't play". Ponieważ dla "humidity" mniejszej niż $ 75\% $ i "outlook" = "sunny" drzewo podejmuje decyzję "Play", taką samą jak przy "outlook" = "overcast", to prtawdopodobieństwa wtedy się sumują i lagorytm podejmuje decyzję "Play" z prawdopodobieństwem $ 100 \% $ .

\end{itemize}

\subsection{Różnica między \emph{gain ratio} a~\emph{info gain} w~praktyce}

\begin{itemize}
\item Obejrzyj zbiór \emph{testgain} (\emph{dat} i~\emph{nam}).
\item Wygeneruj dla niego dwukrotnie drzewo z~użyciem opcji \emph{gain ratio} i~\emph{info gain}; skomentuj wyniki.
	\\

	Drzewo wygenerowane z użyciem opcji InfoGain:		
		\begin{verbatim}
			a1 = 1: A (2.0)
			a1 = 2: A (2.0)
			a1 = 3: B (2.0)
			a1 = 4: B (2.0)	
		\end{verbatim}


	Drzewo wygenerowane z użyciem opcji GainRatio:
		\begin{verbatim}
			a2 = 1: A (4.0)
			a2 = 2: B (4.0)
		\end{verbatim}

	Oba atrybuty są typu wyliczeniowego (pzyjmują wartości z określonego w definicji problemu zbioru). Zbiór wartości atrybutu a1 ma wielkość 4 a atrybutu a2 wielkość 2. Współczynnik InfoGain dla obu atrybutów wynosi 1, nie preferuje więc atrybutu a2, którego dziedzina ma mniej  wartości niż dziedzina atrybutu a1. Inaczej jest w przypadku współczynnika GainRatio:  preferuje on atrybuty o mniejszych dziedzinach. Współczynnik ten wynosi $ 1/2  $ dla atrybutu a1 i $ 1 $ dla atrybytu a2, co za tym idzie w drzewie zbudowanym przy jego pomocy brany jest pod uwagę atrybut a2. Jak widać wpływa to pozytywnie na rozmiar drzewa a co za tym idzie jego prostotę i łatwość interpretacji.

	
\end{itemize}

\subsection{Grupowanie wartości atrybutów}

\begin{itemize}
\item Wygeneruj drzewo dla zbioru \emph{testgain}, zaznaczając opcję \emph{Subsetting}.
	\begin{verbatim}
	a1 in {1,2}: A (4.0)
	a1 in {3,4}: B (4.0)
	\end{verbatim}
Jak widać grupowanie wpłynęło pozytywnie na wielkość generowanego drzewa.

\item Analogicznie dla \emph{CRX}: opisać problem (przyznawanie kard kretytowych), obejrzeć zbiór (atrybuty \emph{A4}, \emph{A6} i~\emph{A7} mają wiele wartości); wygenerować drzewo bez i~z~grupowaniem.
\\Ponieważ dane dotyczące tego problemu zawierały poufne dane etykiety atrybutów zostały zmienione na nie nieznaczące symbole.
\\W problemie tym występują dwie klasy decyzyjne: "+" i "-". W zbiorze uczącym mogą mieć one interpretację klientów, którzy odpowiednio: spłacali lub niespłacali zaciągniętych kredytów w terminie spłaty. Przy klasyfikacji mogą mieć one interpretację: pozytywne lub negatywne rozpatrzenie wniosku o przyznanie karty kredytowej.
Każdy klient opisany jest za pomocą 15 atrybutóW, w tym 6 atrybutów przyjmujących wartości ciągłe i 9 atrybutów przyjmujących wartości ze znanych 2 do 14 elementowych zbiorów.



\item Obejrzeć macierz pomyłek dla zbioru uczącego i~testującego; czy w~tym zastosowaniu przydałaby się macierz kosztów pomyłek?
\\Dla drzewa bez grupowania procent błędnych klasyfikacji w zbiorze trenującym wynosi $ 6,5\% $, w zbiorze testowym $20,5\%$.
\\Dla drzewa z grupowaniem wartości te wynoszą odpowiednio $6,7\%$ i $17,5\%$. Na podstawie tych wartości można się domyślać, że występuje zjawisko przeuczenia: dla większego drzewa 
(bez grupowania) różnica w trafność klasyfikacji przykładów ze zbioru uczącego i testowego jest większa niż dla mniejszego drzewa.
\\To czy warto wprowadzić macierz kosztów pomyłek zależy od analiz ekonomicznych banku - może okazać się, że mniejszą stratę przynosi wydanie karty "złemu" klientowi niż niewydanie karty "dobremu" klientowi - np. klient niespłącający w terminie może być wbrew pozorom zjawiskiem korzystnym, ponieważ jeśli w końcu zapłaci karne odsetki to mogą one stanowić dla banku dodatkowy zysk, większy niż koszty obsługi kredytu. Zgodnie z wiedzą banku macierz kosztów pomyłek powinna być skonstruowana tak, żeby minimalizować błędy pierwszego (false positive) lub drugiego (false negative) rodzaju, w zależności od tego które z nich są dla banku bardziej kosztowne.

\end{itemize}

\subsection{Poszukiwanie optymalnej wielkości drzewa uproszczonego}

\begin{itemize}
\item Poszukiwanie optymalnej wielkości drzewa uproszczonego przez dobór poziomu ufności procedury upraszczającej (\emph{Pruning confidence level}); przeprowadź serię eksperymentów \emph{10-fold cross-validation} dla zbioru \emph{Monk2}, ze~zmieniającym się poziomem ufności od~0.05 do~0.5, z~krokiem co najwyżej 0.05. Sporządź wykres zależności:

\begin{itemize}
\item średniego (po \emph{cross-validation}) rozmiaru drzewa uproszczonego,
\\ % GNUPLOT: LaTeX picture
\setlength{\unitlength}{0.240900pt}
\ifx\plotpoint\undefined\newsavebox{\plotpoint}\fi
\begin{picture}(1500,900)(0,0)
\sbox{\plotpoint}{\rule[-0.200pt]{0.400pt}{0.400pt}}%
\put(110.0,82.0){\rule[-0.200pt]{4.818pt}{0.400pt}}
\put(90,82){\makebox(0,0)[r]{ 5}}
\put(1419.0,82.0){\rule[-0.200pt]{4.818pt}{0.400pt}}
\put(110.0,193.0){\rule[-0.200pt]{4.818pt}{0.400pt}}
\put(90,193){\makebox(0,0)[r]{ 10}}
\put(1419.0,193.0){\rule[-0.200pt]{4.818pt}{0.400pt}}
\put(110.0,304.0){\rule[-0.200pt]{4.818pt}{0.400pt}}
\put(90,304){\makebox(0,0)[r]{ 15}}
\put(1419.0,304.0){\rule[-0.200pt]{4.818pt}{0.400pt}}
\put(110.0,415.0){\rule[-0.200pt]{4.818pt}{0.400pt}}
\put(90,415){\makebox(0,0)[r]{ 20}}
\put(1419.0,415.0){\rule[-0.200pt]{4.818pt}{0.400pt}}
\put(110.0,526.0){\rule[-0.200pt]{4.818pt}{0.400pt}}
\put(90,526){\makebox(0,0)[r]{ 25}}
\put(1419.0,526.0){\rule[-0.200pt]{4.818pt}{0.400pt}}
\put(110.0,637.0){\rule[-0.200pt]{4.818pt}{0.400pt}}
\put(90,637){\makebox(0,0)[r]{ 30}}
\put(1419.0,637.0){\rule[-0.200pt]{4.818pt}{0.400pt}}
\put(110.0,748.0){\rule[-0.200pt]{4.818pt}{0.400pt}}
\put(90,748){\makebox(0,0)[r]{ 35}}
\put(1419.0,748.0){\rule[-0.200pt]{4.818pt}{0.400pt}}
\put(110.0,859.0){\rule[-0.200pt]{4.818pt}{0.400pt}}
\put(90,859){\makebox(0,0)[r]{ 40}}
\put(1419.0,859.0){\rule[-0.200pt]{4.818pt}{0.400pt}}
\put(110.0,82.0){\rule[-0.200pt]{0.400pt}{4.818pt}}
\put(110,41){\makebox(0,0){ 0.05}}
\put(110.0,839.0){\rule[-0.200pt]{0.400pt}{4.818pt}}
\put(258.0,82.0){\rule[-0.200pt]{0.400pt}{4.818pt}}
\put(258,41){\makebox(0,0){ 0.1}}
\put(258.0,839.0){\rule[-0.200pt]{0.400pt}{4.818pt}}
\put(405.0,82.0){\rule[-0.200pt]{0.400pt}{4.818pt}}
\put(405,41){\makebox(0,0){ 0.15}}
\put(405.0,839.0){\rule[-0.200pt]{0.400pt}{4.818pt}}
\put(553.0,82.0){\rule[-0.200pt]{0.400pt}{4.818pt}}
\put(553,41){\makebox(0,0){ 0.2}}
\put(553.0,839.0){\rule[-0.200pt]{0.400pt}{4.818pt}}
\put(701.0,82.0){\rule[-0.200pt]{0.400pt}{4.818pt}}
\put(701,41){\makebox(0,0){ 0.25}}
\put(701.0,839.0){\rule[-0.200pt]{0.400pt}{4.818pt}}
\put(848.0,82.0){\rule[-0.200pt]{0.400pt}{4.818pt}}
\put(848,41){\makebox(0,0){ 0.3}}
\put(848.0,839.0){\rule[-0.200pt]{0.400pt}{4.818pt}}
\put(996.0,82.0){\rule[-0.200pt]{0.400pt}{4.818pt}}
\put(996,41){\makebox(0,0){ 0.35}}
\put(996.0,839.0){\rule[-0.200pt]{0.400pt}{4.818pt}}
\put(1144.0,82.0){\rule[-0.200pt]{0.400pt}{4.818pt}}
\put(1144,41){\makebox(0,0){ 0.4}}
\put(1144.0,839.0){\rule[-0.200pt]{0.400pt}{4.818pt}}
\put(1291.0,82.0){\rule[-0.200pt]{0.400pt}{4.818pt}}
\put(1291,41){\makebox(0,0){ 0.45}}
\put(1291.0,839.0){\rule[-0.200pt]{0.400pt}{4.818pt}}
\put(1439.0,82.0){\rule[-0.200pt]{0.400pt}{4.818pt}}
\put(1439,41){\makebox(0,0){ 0.5}}
\put(1439.0,839.0){\rule[-0.200pt]{0.400pt}{4.818pt}}
\put(110.0,82.0){\rule[-0.200pt]{0.400pt}{187.179pt}}
\put(110.0,82.0){\rule[-0.200pt]{320.156pt}{0.400pt}}
\put(1439.0,82.0){\rule[-0.200pt]{0.400pt}{187.179pt}}
\put(110.0,859.0){\rule[-0.200pt]{320.156pt}{0.400pt}}
\put(1279,819){\makebox(0,0)[r]{"crosval.dat" using 1:2}}
\put(1299.0,819.0){\rule[-0.200pt]{24.090pt}{0.400pt}}
\put(110,184){\usebox{\plotpoint}}
\multiput(110.58,184.00)(0.499,0.547){293}{\rule{0.120pt}{0.538pt}}
\multiput(109.17,184.00)(148.000,160.884){2}{\rule{0.400pt}{0.269pt}}
\multiput(405.00,346.58)(5.849,0.493){23}{\rule{4.654pt}{0.119pt}}
\multiput(405.00,345.17)(138.341,13.000){2}{\rule{2.327pt}{0.400pt}}
\multiput(553.00,359.58)(0.788,0.499){185}{\rule{0.730pt}{0.120pt}}
\multiput(553.00,358.17)(146.485,94.000){2}{\rule{0.365pt}{0.400pt}}
\multiput(701.00,453.58)(1.189,0.499){121}{\rule{1.048pt}{0.120pt}}
\multiput(701.00,452.17)(144.824,62.000){2}{\rule{0.524pt}{0.400pt}}
\multiput(848.00,515.58)(1.582,0.498){91}{\rule{1.360pt}{0.120pt}}
\multiput(848.00,514.17)(145.178,47.000){2}{\rule{0.680pt}{0.400pt}}
\multiput(996.00,562.58)(0.587,0.499){249}{\rule{0.570pt}{0.120pt}}
\multiput(996.00,561.17)(146.817,126.000){2}{\rule{0.285pt}{0.400pt}}
\multiput(1144.00,688.58)(1.271,0.499){113}{\rule{1.114pt}{0.120pt}}
\multiput(1144.00,687.17)(144.688,58.000){2}{\rule{0.557pt}{0.400pt}}
\multiput(1291.00,746.58)(1.401,0.498){103}{\rule{1.217pt}{0.120pt}}
\multiput(1291.00,745.17)(145.474,53.000){2}{\rule{0.608pt}{0.400pt}}
\put(110,184){\makebox(0,0){$+$}}
\put(258,346){\makebox(0,0){$+$}}
\put(405,346){\makebox(0,0){$+$}}
\put(553,359){\makebox(0,0){$+$}}
\put(701,453){\makebox(0,0){$+$}}
\put(848,515){\makebox(0,0){$+$}}
\put(996,562){\makebox(0,0){$+$}}
\put(1144,688){\makebox(0,0){$+$}}
\put(1291,746){\makebox(0,0){$+$}}
\put(1439,799){\makebox(0,0){$+$}}
\put(1349,819){\makebox(0,0){$+$}}
\put(258.0,346.0){\rule[-0.200pt]{35.412pt}{0.400pt}}
\put(110.0,82.0){\rule[-0.200pt]{0.400pt}{187.179pt}}
\put(110.0,82.0){\rule[-0.200pt]{320.156pt}{0.400pt}}
\put(1439.0,82.0){\rule[-0.200pt]{0.400pt}{187.179pt}}
\put(110.0,859.0){\rule[-0.200pt]{320.156pt}{0.400pt}}
\end{picture}

\item średniej trafności klasyfikowania drzewa uproszczonego na~zbiorze testującym,
\\ % GNUPLOT: LaTeX picture
\setlength{\unitlength}{0.240900pt}
\ifx\plotpoint\undefined\newsavebox{\plotpoint}\fi
\begin{picture}(1500,900)(0,0)
\sbox{\plotpoint}{\rule[-0.200pt]{0.400pt}{0.400pt}}%
\put(191.0,131.0){\rule[-0.200pt]{4.818pt}{0.400pt}}
\put(171,131){\makebox(0,0)[r]{ 0.56}}
\put(1419.0,131.0){\rule[-0.200pt]{4.818pt}{0.400pt}}
\put(191.0,235.0){\rule[-0.200pt]{4.818pt}{0.400pt}}
\put(171,235){\makebox(0,0)[r]{ 0.57}}
\put(1419.0,235.0){\rule[-0.200pt]{4.818pt}{0.400pt}}
\put(191.0,339.0){\rule[-0.200pt]{4.818pt}{0.400pt}}
\put(171,339){\makebox(0,0)[r]{ 0.58}}
\put(1419.0,339.0){\rule[-0.200pt]{4.818pt}{0.400pt}}
\put(191.0,443.0){\rule[-0.200pt]{4.818pt}{0.400pt}}
\put(171,443){\makebox(0,0)[r]{ 0.59}}
\put(1419.0,443.0){\rule[-0.200pt]{4.818pt}{0.400pt}}
\put(191.0,547.0){\rule[-0.200pt]{4.818pt}{0.400pt}}
\put(171,547){\makebox(0,0)[r]{ 0.6}}
\put(1419.0,547.0){\rule[-0.200pt]{4.818pt}{0.400pt}}
\put(191.0,651.0){\rule[-0.200pt]{4.818pt}{0.400pt}}
\put(171,651){\makebox(0,0)[r]{ 0.61}}
\put(1419.0,651.0){\rule[-0.200pt]{4.818pt}{0.400pt}}
\put(191.0,755.0){\rule[-0.200pt]{4.818pt}{0.400pt}}
\put(171,755){\makebox(0,0)[r]{ 0.62}}
\put(1419.0,755.0){\rule[-0.200pt]{4.818pt}{0.400pt}}
\put(191.0,859.0){\rule[-0.200pt]{4.818pt}{0.400pt}}
\put(171,859){\makebox(0,0)[r]{ 0.63}}
\put(1419.0,859.0){\rule[-0.200pt]{4.818pt}{0.400pt}}
\put(191.0,131.0){\rule[-0.200pt]{0.400pt}{4.818pt}}
\put(191,90){\makebox(0,0){ 0.05}}
\put(191.0,839.0){\rule[-0.200pt]{0.400pt}{4.818pt}}
\put(330.0,131.0){\rule[-0.200pt]{0.400pt}{4.818pt}}
\put(330,90){\makebox(0,0){ 0.1}}
\put(330.0,839.0){\rule[-0.200pt]{0.400pt}{4.818pt}}
\put(468.0,131.0){\rule[-0.200pt]{0.400pt}{4.818pt}}
\put(468,90){\makebox(0,0){ 0.15}}
\put(468.0,839.0){\rule[-0.200pt]{0.400pt}{4.818pt}}
\put(607.0,131.0){\rule[-0.200pt]{0.400pt}{4.818pt}}
\put(607,90){\makebox(0,0){ 0.2}}
\put(607.0,839.0){\rule[-0.200pt]{0.400pt}{4.818pt}}
\put(746.0,131.0){\rule[-0.200pt]{0.400pt}{4.818pt}}
\put(746,90){\makebox(0,0){ 0.25}}
\put(746.0,839.0){\rule[-0.200pt]{0.400pt}{4.818pt}}
\put(884.0,131.0){\rule[-0.200pt]{0.400pt}{4.818pt}}
\put(884,90){\makebox(0,0){ 0.3}}
\put(884.0,839.0){\rule[-0.200pt]{0.400pt}{4.818pt}}
\put(1023.0,131.0){\rule[-0.200pt]{0.400pt}{4.818pt}}
\put(1023,90){\makebox(0,0){ 0.35}}
\put(1023.0,839.0){\rule[-0.200pt]{0.400pt}{4.818pt}}
\put(1162.0,131.0){\rule[-0.200pt]{0.400pt}{4.818pt}}
\put(1162,90){\makebox(0,0){ 0.4}}
\put(1162.0,839.0){\rule[-0.200pt]{0.400pt}{4.818pt}}
\put(1300.0,131.0){\rule[-0.200pt]{0.400pt}{4.818pt}}
\put(1300,90){\makebox(0,0){ 0.45}}
\put(1300.0,839.0){\rule[-0.200pt]{0.400pt}{4.818pt}}
\put(1439.0,131.0){\rule[-0.200pt]{0.400pt}{4.818pt}}
\put(1439,90){\makebox(0,0){ 0.5}}
\put(1439.0,839.0){\rule[-0.200pt]{0.400pt}{4.818pt}}
\put(191.0,131.0){\rule[-0.200pt]{0.400pt}{175.375pt}}
\put(191.0,131.0){\rule[-0.200pt]{300.643pt}{0.400pt}}
\put(1439.0,131.0){\rule[-0.200pt]{0.400pt}{175.375pt}}
\put(191.0,859.0){\rule[-0.200pt]{300.643pt}{0.400pt}}
\put(30,495){\makebox(0,0){Trafno��}}
\put(815,29){\makebox(0,0){Pruning confidence level}}
\put(191,589){\usebox{\plotpoint}}
\multiput(191.00,589.58)(1.124,0.499){121}{\rule{0.997pt}{0.120pt}}
\multiput(191.00,588.17)(136.931,62.000){2}{\rule{0.498pt}{0.400pt}}
\multiput(468.00,651.58)(0.610,0.499){225}{\rule{0.588pt}{0.120pt}}
\multiput(468.00,650.17)(137.780,114.000){2}{\rule{0.294pt}{0.400pt}}
\multiput(607.58,760.24)(0.499,-1.311){275}{\rule{0.120pt}{1.147pt}}
\multiput(606.17,762.62)(139.000,-361.618){2}{\rule{0.400pt}{0.574pt}}
\multiput(746.00,399.92)(0.556,-0.499){245}{\rule{0.545pt}{0.120pt}}
\multiput(746.00,400.17)(136.868,-124.000){2}{\rule{0.273pt}{0.400pt}}
\multiput(884.00,275.92)(0.556,-0.499){247}{\rule{0.545pt}{0.120pt}}
\multiput(884.00,276.17)(137.869,-125.000){2}{\rule{0.272pt}{0.400pt}}
\multiput(1023.00,152.58)(0.556,0.499){247}{\rule{0.545pt}{0.120pt}}
\multiput(1023.00,151.17)(137.869,125.000){2}{\rule{0.272pt}{0.400pt}}
\put(330.0,651.0){\rule[-0.200pt]{33.244pt}{0.400pt}}
\multiput(1300.00,275.92)(1.106,-0.499){123}{\rule{0.983pt}{0.120pt}}
\multiput(1300.00,276.17)(136.961,-63.000){2}{\rule{0.491pt}{0.400pt}}
\put(191,589){\makebox(0,0){$+$}}
\put(330,651){\makebox(0,0){$+$}}
\put(468,651){\makebox(0,0){$+$}}
\put(607,765){\makebox(0,0){$+$}}
\put(746,401){\makebox(0,0){$+$}}
\put(884,277){\makebox(0,0){$+$}}
\put(1023,152){\makebox(0,0){$+$}}
\put(1162,277){\makebox(0,0){$+$}}
\put(1300,277){\makebox(0,0){$+$}}
\put(1439,214){\makebox(0,0){$+$}}
\put(1162.0,277.0){\rule[-0.200pt]{33.244pt}{0.400pt}}
\put(191.0,131.0){\rule[-0.200pt]{0.400pt}{175.375pt}}
\put(191.0,131.0){\rule[-0.200pt]{300.643pt}{0.400pt}}
\put(1439.0,131.0){\rule[-0.200pt]{0.400pt}{175.375pt}}
\put(191.0,859.0){\rule[-0.200pt]{300.643pt}{0.400pt}}
\end{picture}

\\ \input{figures/part1/task5/file.tex}
\item średniej estymaty błędu dla drzewa uproszczonego
\\ % GNUPLOT: LaTeX picture
\setlength{\unitlength}{0.240900pt}
\ifx\plotpoint\undefined\newsavebox{\plotpoint}\fi
\begin{picture}(1500,900)(0,0)
\sbox{\plotpoint}{\rule[-0.200pt]{0.400pt}{0.400pt}}%
\put(170.0,82.0){\rule[-0.200pt]{4.818pt}{0.400pt}}
\put(150,82){\makebox(0,0)[r]{ 0.36}}
\put(1419.0,82.0){\rule[-0.200pt]{4.818pt}{0.400pt}}
\put(170.0,168.0){\rule[-0.200pt]{4.818pt}{0.400pt}}
\put(150,168){\makebox(0,0)[r]{ 0.365}}
\put(1419.0,168.0){\rule[-0.200pt]{4.818pt}{0.400pt}}
\put(170.0,255.0){\rule[-0.200pt]{4.818pt}{0.400pt}}
\put(150,255){\makebox(0,0)[r]{ 0.37}}
\put(1419.0,255.0){\rule[-0.200pt]{4.818pt}{0.400pt}}
\put(170.0,341.0){\rule[-0.200pt]{4.818pt}{0.400pt}}
\put(150,341){\makebox(0,0)[r]{ 0.375}}
\put(1419.0,341.0){\rule[-0.200pt]{4.818pt}{0.400pt}}
\put(170.0,427.0){\rule[-0.200pt]{4.818pt}{0.400pt}}
\put(150,427){\makebox(0,0)[r]{ 0.38}}
\put(1419.0,427.0){\rule[-0.200pt]{4.818pt}{0.400pt}}
\put(170.0,514.0){\rule[-0.200pt]{4.818pt}{0.400pt}}
\put(150,514){\makebox(0,0)[r]{ 0.385}}
\put(1419.0,514.0){\rule[-0.200pt]{4.818pt}{0.400pt}}
\put(170.0,600.0){\rule[-0.200pt]{4.818pt}{0.400pt}}
\put(150,600){\makebox(0,0)[r]{ 0.39}}
\put(1419.0,600.0){\rule[-0.200pt]{4.818pt}{0.400pt}}
\put(170.0,686.0){\rule[-0.200pt]{4.818pt}{0.400pt}}
\put(150,686){\makebox(0,0)[r]{ 0.395}}
\put(1419.0,686.0){\rule[-0.200pt]{4.818pt}{0.400pt}}
\put(170.0,773.0){\rule[-0.200pt]{4.818pt}{0.400pt}}
\put(150,773){\makebox(0,0)[r]{ 0.4}}
\put(1419.0,773.0){\rule[-0.200pt]{4.818pt}{0.400pt}}
\put(170.0,859.0){\rule[-0.200pt]{4.818pt}{0.400pt}}
\put(150,859){\makebox(0,0)[r]{ 0.405}}
\put(1419.0,859.0){\rule[-0.200pt]{4.818pt}{0.400pt}}
\put(170.0,82.0){\rule[-0.200pt]{0.400pt}{4.818pt}}
\put(170,41){\makebox(0,0){ 0.05}}
\put(170.0,839.0){\rule[-0.200pt]{0.400pt}{4.818pt}}
\put(311.0,82.0){\rule[-0.200pt]{0.400pt}{4.818pt}}
\put(311,41){\makebox(0,0){ 0.1}}
\put(311.0,839.0){\rule[-0.200pt]{0.400pt}{4.818pt}}
\put(452.0,82.0){\rule[-0.200pt]{0.400pt}{4.818pt}}
\put(452,41){\makebox(0,0){ 0.15}}
\put(452.0,839.0){\rule[-0.200pt]{0.400pt}{4.818pt}}
\put(593.0,82.0){\rule[-0.200pt]{0.400pt}{4.818pt}}
\put(593,41){\makebox(0,0){ 0.2}}
\put(593.0,839.0){\rule[-0.200pt]{0.400pt}{4.818pt}}
\put(734.0,82.0){\rule[-0.200pt]{0.400pt}{4.818pt}}
\put(734,41){\makebox(0,0){ 0.25}}
\put(734.0,839.0){\rule[-0.200pt]{0.400pt}{4.818pt}}
\put(875.0,82.0){\rule[-0.200pt]{0.400pt}{4.818pt}}
\put(875,41){\makebox(0,0){ 0.3}}
\put(875.0,839.0){\rule[-0.200pt]{0.400pt}{4.818pt}}
\put(1016.0,82.0){\rule[-0.200pt]{0.400pt}{4.818pt}}
\put(1016,41){\makebox(0,0){ 0.35}}
\put(1016.0,839.0){\rule[-0.200pt]{0.400pt}{4.818pt}}
\put(1157.0,82.0){\rule[-0.200pt]{0.400pt}{4.818pt}}
\put(1157,41){\makebox(0,0){ 0.4}}
\put(1157.0,839.0){\rule[-0.200pt]{0.400pt}{4.818pt}}
\put(1298.0,82.0){\rule[-0.200pt]{0.400pt}{4.818pt}}
\put(1298,41){\makebox(0,0){ 0.45}}
\put(1298.0,839.0){\rule[-0.200pt]{0.400pt}{4.818pt}}
\put(1439.0,82.0){\rule[-0.200pt]{0.400pt}{4.818pt}}
\put(1439,41){\makebox(0,0){ 0.5}}
\put(1439.0,839.0){\rule[-0.200pt]{0.400pt}{4.818pt}}
\put(170.0,82.0){\rule[-0.200pt]{0.400pt}{187.179pt}}
\put(170.0,82.0){\rule[-0.200pt]{305.702pt}{0.400pt}}
\put(1439.0,82.0){\rule[-0.200pt]{0.400pt}{187.179pt}}
\put(170.0,859.0){\rule[-0.200pt]{305.702pt}{0.400pt}}
\put(1279,819){\makebox(0,0)[r]{"crosval.dat" using 1:4}}
\put(1299.0,819.0){\rule[-0.200pt]{24.090pt}{0.400pt}}
\put(170,807){\usebox{\plotpoint}}
\multiput(170.00,805.92)(1.024,-0.499){135}{\rule{0.917pt}{0.120pt}}
\multiput(170.00,806.17)(139.096,-69.000){2}{\rule{0.459pt}{0.400pt}}
\multiput(311.00,736.92)(1.361,-0.498){101}{\rule{1.185pt}{0.120pt}}
\multiput(311.00,737.17)(138.541,-52.000){2}{\rule{0.592pt}{0.400pt}}
\multiput(452.00,684.92)(1.388,-0.498){99}{\rule{1.206pt}{0.120pt}}
\multiput(452.00,685.17)(138.497,-51.000){2}{\rule{0.603pt}{0.400pt}}
\multiput(593.00,633.92)(1.009,-0.499){137}{\rule{0.906pt}{0.120pt}}
\multiput(593.00,634.17)(139.120,-70.000){2}{\rule{0.453pt}{0.400pt}}
\multiput(734.00,563.92)(1.024,-0.499){135}{\rule{0.917pt}{0.120pt}}
\multiput(734.00,564.17)(139.096,-69.000){2}{\rule{0.459pt}{0.400pt}}
\multiput(875.00,494.92)(1.024,-0.499){135}{\rule{0.917pt}{0.120pt}}
\multiput(875.00,495.17)(139.096,-69.000){2}{\rule{0.459pt}{0.400pt}}
\multiput(1016.00,425.92)(0.685,-0.499){203}{\rule{0.648pt}{0.120pt}}
\multiput(1016.00,426.17)(139.656,-103.000){2}{\rule{0.324pt}{0.400pt}}
\multiput(1157.00,322.92)(0.583,-0.499){239}{\rule{0.566pt}{0.120pt}}
\multiput(1157.00,323.17)(139.825,-121.000){2}{\rule{0.283pt}{0.400pt}}
\multiput(1298.00,201.92)(0.678,-0.499){205}{\rule{0.642pt}{0.120pt}}
\multiput(1298.00,202.17)(139.667,-104.000){2}{\rule{0.321pt}{0.400pt}}
\put(170,807){\makebox(0,0){$+$}}
\put(311,738){\makebox(0,0){$+$}}
\put(452,686){\makebox(0,0){$+$}}
\put(593,635){\makebox(0,0){$+$}}
\put(734,565){\makebox(0,0){$+$}}
\put(875,496){\makebox(0,0){$+$}}
\put(1016,427){\makebox(0,0){$+$}}
\put(1157,324){\makebox(0,0){$+$}}
\put(1298,203){\makebox(0,0){$+$}}
\put(1439,99){\makebox(0,0){$+$}}
\put(1349,819){\makebox(0,0){$+$}}
\put(170.0,82.0){\rule[-0.200pt]{0.400pt}{187.179pt}}
\put(170.0,82.0){\rule[-0.200pt]{305.702pt}{0.400pt}}
\put(1439.0,82.0){\rule[-0.200pt]{0.400pt}{187.179pt}}
\put(170.0,859.0){\rule[-0.200pt]{305.702pt}{0.400pt}}
\end{picture}

\end{itemize}

...w funkcji poziomu ufności (odnieś te wyniki do średniej charakterystyki drzewa oryginalnego, nieuproszczonego).




\item Poszukiwanie optymalnej wielkości drzewa uproszczonego poprzez prepruning, tj.~manewrowanie minimalną licznością węzła (\emph{Minimum objects}). Dla zbioru \emph{CRX} przebadać przedział od~2 do~10.
\item Analiza wygenerowanego drzewa: poszukiwanie słabych punktów (liści o~małym wsparciu, poddrzew które generują szczególnie dużo błędów, etc.).


\end{itemize}

\subsection{\emph{Windowing}}

\begin{itemize}
\item Wyjaśnić zasadę i~opcje: \emph{Trials}, \emph{Initial window size}, \emph{Window increment}.
\item Analiza wyników (\emph{CRX}).
\end{itemize}

\subsection{Generowanie krzywej uczenia}

\begin{itemize}
\item Dla zbioru \emph{vote} przygotować kilka[naście] zbiorów uczących o~liczności $n$ zmieniającej się od~50 do~300, ze~skokiem np.~50 przypadków, poprzez wybieranie pierwszych $n$ ze~zbioru \emph{vote.dat}.
\item Wykreślić jako funkcję $n$ rozmiar drzewa uproszczonego oraz trafność klasyfikowania drzewa uproszczonego na zbiorze testującym.

\end{itemize}

\subsection{Maksymalizacja trafności}

\begin{itemize}
\item Uzyskaj jak najwyższą trafność klasyfikowania ze~zbioru \emph{GERMAN} w~eksperymencie \emph{10-fold CV}. Jakimi parametrami i~mechanizmami można manipulować, by~szukać najwyższej trafności? Kiedy można ufać tak uzyskanej trafności, a~kiedy można mówić o~nadużyciu?
\end{itemize}