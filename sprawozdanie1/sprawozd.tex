\documentclass{article}
\usepackage{polski} %moze wymagac dokonfigurowania latexa, ale jest lepszy niż standardowy babel'owy [polish] 
\usepackage[utf8]{inputenc} 
\usepackage[OT4]{fontenc} 
\usepackage{graphicx,color} %include pdf's (and png's for raster graphics... avoid raster graphics!) 
\usepackage{url} 
\usepackage[pdftex,hyperfootnotes=false,pdfborder={0 0 0}]{hyperref} %za wszystkimi pakietami; pdfborder nie wszedzie tak samo zaimplementowane bo specyfikacja nieprecyzyjna; pod miktex'em po prostu nie widac wtedy ramek


% Zmiana rozmiar�w strony tekstu
\addtolength{\voffset}{-1cm}
\addtolength{\hoffset}{-1cm}
\addtolength{\textwidth}{2cm}
\addtolength{\textheight}{2cm}

%bardziej zyciowe parametry sterujace rozmieszczeniem rysunkow
\renewcommand{\topfraction}{.85}
\renewcommand{\bottomfraction}{.7}
\renewcommand{\textfraction}{.15}
\renewcommand{\floatpagefraction}{.66}
\renewcommand{\dbltopfraction}{.66}
\renewcommand{\dblfloatpagefraction}{.66}
\setcounter{topnumber}{9}
\setcounter{bottomnumber}{9}
\setcounter{totalnumber}{20}
\setcounter{dbltopnumber}{9}

% w�asny bullet list z malymi odstepami
\newenvironment{tightlist}{
\begin{itemize}
  \setlength{\itemsep}{1pt}
  \setlength{\parskip}{0pt}
  \setlength{\parsep}{0pt}}
{\end{itemize}}




\begin{document}

\thispagestyle{empty} %bez numeru strony

\begin{center}
{\large{Sprawozdanie z laboratorium:\\
Uczenie MAszynowe i Sieci Neuronowe}}

\vspace{3ex}

Część I: Generowanie drzew decyzyjnych
Część II: Generowanie reguł decyzyjnych


\vspace{3ex}
{\footnotesize\today}

\end{center}


\vspace{10ex}

Prowadzący: dr inż. Maciej Komosiński

\vspace{5ex}

Autorzy:
\begin{tabular}{lllr}
\textbf{Krzysztof Urban} & inf84896 & ISWD & krz.urb@gmail.com \\
\textbf{Tomasz Ziętkiewicz} & inf84914 & ISWD & tomek.zietkiewicz@gmail.com \\
\end{tabular}

\vspace{5ex}

Krzysztof: zajęcia poniedziałkowe, 16:50
Tomek: zajęcia poniedziałkowe, 13:30


\newpage



\section{Generowanie drzew decyzyjnych}

\subsection{Generowanie drzewa}

\subsection{Konsultowanie}

\subsection{Różnica między gain ratio a info gain w praktyce}

\subsection{Grupowanie wartości atrybutów}

\subsection{Poszukiwanie optymalnej wielkości drzewa uproszczonego}

\subsection{Windowing}

\subsection{Generowanie krzywej uczenia}

\subsection{Maksymalizacja trafności}

\section{Generowanie reguł decyzyjnych}

\subsection{Metoda pośrednia generowania reguł (C4.5rules)}

\subsection{Porównanie klasyfikowania za pomocą drzew decyzyjnych i~reguł decyzyjnych (C4.5rules)}

\subsection{Generowanie reguł z~użyciem algorytmu LEM}

\subsection{Porównanie reguł generowanych za~pomocą algorytmu LEM i~C4.5}

%Dla chętnych:
%\subsection{Przeprowadź eksperyment generowania i~testowania reguł LEM w~ramach cross validation}

%%%%%%%%%%%%%%%% literatura %%%%%%%%%%%%%%%%

\bibliography{sprawozd}
\bibliographystyle{plain}


\end{document}

