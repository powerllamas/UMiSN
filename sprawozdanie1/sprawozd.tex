\documentclass{article}
\usepackage{polski} %moze wymagac dokonfigurowania latexa, ale jest lepszy niż standardowy babel'owy [polish] 
\usepackage[utf8]{inputenc} 
\usepackage[OT4]{fontenc} 
\usepackage{graphicx,color} %include pdf's (and png's for raster graphics... avoid raster graphics!) 
\usepackage{url} 
\usepackage[pdftex,hyperfootnotes=false,pdfborder={0 0 0}]{hyperref} %za wszystkimi pakietami; pdfborder nie wszedzie tak samo zaimplementowane bo specyfikacja nieprecyzyjna; pod miktex'em po prostu nie widac wtedy ramek


% Zmiana rozmiar�w strony tekstu
\addtolength{\voffset}{-1cm}
\addtolength{\hoffset}{-1cm}
\addtolength{\textwidth}{2cm}
\addtolength{\textheight}{2cm}

%bardziej zyciowe parametry sterujace rozmieszczeniem rysunkow
\renewcommand{\topfraction}{.85}
\renewcommand{\bottomfraction}{.7}
\renewcommand{\textfraction}{.15}
\renewcommand{\floatpagefraction}{.66}
\renewcommand{\dbltopfraction}{.66}
\renewcommand{\dblfloatpagefraction}{.66}
\setcounter{topnumber}{9}
\setcounter{bottomnumber}{9}
\setcounter{totalnumber}{20}
\setcounter{dbltopnumber}{9}

% w�asny bullet list z malymi odstepami
\newenvironment{tightlist}{
\begin{itemize}
  \setlength{\itemsep}{1pt}
  \setlength{\parskip}{0pt}
  \setlength{\parsep}{0pt}}
{\end{itemize}}




\begin{document}

\thispagestyle{empty} %bez numeru strony

\begin{center}
{\large{Sprawozdanie II z laboratorium:\\
Uczenie Maszynowe i Sieci Neuronowe}}

\vspace{3ex}

Część I: Uczenie nadzorowane warstwowych sieci neuronowych

Część II: Rekurencyjne sieci neuronowe

Część III: Uczenie nienadzorowane warstwowych sieci neuronowych


\vspace{3ex}
{\footnotesize\today}

\end{center}


\vspace{10ex}

Prowadzący: dr inż. Maciej Komosiński

\vspace{5ex}

Autorzy:
\begin{tabular}{lllr}
\textbf{Krzysztof Urban} & inf84896 & ISWD & krz.urb@gmail.com \\
\textbf{Tomasz Ziętkiewicz} & inf84914 & ISWD & tomek.zietkiewicz@gmail.com \\
\end{tabular}

\vspace{5ex}

Zajęcia poniedziałkowe; Krzysztof: 16:50, Tomasz: 13:30

\newpage



%%%%%%%%%%%%%%%%%% Część I %%%%%%%%%%%%%%%%%

\section{Generowanie drzew decyzyjnych}

\subsection{Generowanie drzewa}

\subsection{Konsultowanie}

\subsection{Różnica między \emph{gain ratio} a~\emph{info gain} w~praktyce}

\subsection{Grupowanie wartości atrybutów}

\subsection{Poszukiwanie optymalnej wielkości drzewa uproszczonego}

\subsection{\emph{Windowing}}

\subsection{Generowanie krzywej uczenia}

\subsection{Maksymalizacja trafności}

%%%%%%%%%%%%%%%%%% Część II %%%%%%%%%%%%%%%%%

\section{Generowanie reguł decyzyjnych}

\subsection{Metoda pośrednia generowania reguł (\emph{C4.5rules})}

\subsubsection{Wygeneruj reguły dla zbioru \emph{GOLF} za~pomocą programu \emph{C4.5 for Windows}.}
\subsubsection{Porównaj wygenerowane reguły z~wyjściowym drzewem decyzyjnym. Czy reguły odzwierciedlają precyzyjnie drzewo?}

\subsection{Porównanie klasyfikowania za pomocą drzew decyzyjnych i~reguł decyzyjnych (\emph{C4.5rules})}

\begin{itemize}
\item Przeprowadź testy \emph{10-fold CV} na wybranych zbiorach dla drzew i~reguł.
\item Porównaj wyniki pod kątem trafności klasyfikowania na zbiorze testującym oraz rozmiaru opisu.
\item Przeprowadzając kilka eksperymentów uczenia i~testowania przeanalizuj wpływ parametrów \emph{Confidence Level} i~\emph{Redundancy Factor} na~otrzymywany zbiór reguł.
\end{itemize}

\subsection{Generowanie reguł z~użyciem algorytmu \emph{LEM}}

\begin{itemize}
\item Wygeneruj reguły dla zbioru \emph{HPAP.ISF}.
\item Przyjrzyj się regułom możliwym; opisz je i~,,wydedukuj'', skąd się wzięły.
\end{itemize}

\subsection{Porównanie reguł generowanych za~pomocą algorytmu \emph{LEM} i~\emph{C4.5}}

\begin{itemize}
\item Wygeneruj reguły przy użyciu obu podejść dla zbiorów: \emph{HPAP}, \emph{VOTE} i~\emph{MONK}.
\item Przyjrzyj się niezależnie regułom pewnym i~możliwym (\emph{LEM}).
\end{itemize}

%Dla chętnych:
%\subsection{Przeprowadź eksperyment generowania i~testowania reguł LEM w~ramach cross validation}

%%%%%%%%%%%%%%%% literatura %%%%%%%%%%%%%%%%

\bibliography{sprawozd}
\bibliographystyle{plain}


\end{document}

